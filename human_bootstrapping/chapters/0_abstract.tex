\section*{Executive Summary}

This memorandum details how the current treatment of mail from U.S. military post offices (APO/FPO/DPO) as "international" constitutes a multi-faceted violation of the U.S. Constitution. By subjecting the personal effects of U.S. service members and citizens to customs scrutiny, the U.S. government has created a system that:
\begin{itemize}
\item Violates the Fourth Amendment right to be free from unreasonable searches and seizures.
\item Denies Fifth Amendment Due Process and Equal Protection by creating a second-tier class of citizens.
\item Cedes U.S. sovereignty via Status of Forces Agreements (SOFAs) in a manner that infringes upon fundamental rights, an unconstitutional act by the Executive and Legislative branches.
\end{itemize}

We demand the USPIS and Congress take corrective action to reclassify such mail as domestic, thereby restoring full constitutional protections to Americans serving abroad.

As of August 28, 2025, recent developments, including enhanced enforcement of customs declaration forms for military and diplomatic mail (effective from 2024 onwards) and new size restrictions for DPO packages (effective July 18, 2025), underscore the ongoing nature of these violations without substantive reform \cite{usps2024enhanced} \cite{dod2025dpo}. Despite these updates, the core practice of treating APO/FPO/DPO mail as international persists, requiring customs forms like PS Form 2976 for most shipments, in direct conflict with constitutional principles \cite{usps2023imm} \cite{cbp2023guide}.
