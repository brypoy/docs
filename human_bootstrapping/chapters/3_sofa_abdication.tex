\section{The Unconstitutional Abdication of Sovereignty via SOFAs}

\subsection{Constitutional Claim}
Article VI establishes the Constitution as the "supreme Law of the Land" \cite{usconstitution}. The Executive Branch (via treaty-making) and Legislative Branch (via ratification) cannot enter into agreements that violate the fundamental rights of citizens.

\subsection{Contradictory Law and Practice}
Status of Forces Agreements (SOFAs) that concede host-nation jurisdiction over the movement of goods from U.S. bases, necessitating their treatment as "exports," force the violation of citizens' Fourth and Fifth Amendment rights.

\subsection{Legal Argument}
While the President and Congress have the power to make treaties under Article II, Section 2, this power is \textbf{not unlimited}. It cannot be used to circumvent the Bill of Rights or abdicate the government's primary duty to protect its citizens' constitutional rights.

A SOFA that forces the U.S. government to treat its own military installations as foreign territory for customs purposes represents an \textbf{unconstitutional abdication of sovereignty}. The U.S. government's primary duty is to its citizens, not to the convenience of a diplomatic agreement that strips Americans of their constitutional protections.

The current interpretation and application of SOFAs creates a situation where the U.S. government acts as an agent of foreign powers in violating the rights of its own citizens, a clear violation of constitutional principles.

\subsection{Supporting Case Law}
\begin{itemize}
    \item \textbf{Reid v. Covert (1957)} \cite{reid1957}: Explicitly states that "no agreement with a foreign nation can confer power on the Congress, or on any other branch of Government, which is free from the restraints of the Constitution."
    \item \textbf{Missouri v. Holland (1920)} \cite{missouri1920}: While affirming treaty power, this case acknowledged constitutional limitations on such power.
\end{itemize}
